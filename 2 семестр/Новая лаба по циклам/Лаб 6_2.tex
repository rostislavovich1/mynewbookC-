\documentclass[a4paper]{article}
\usepackage[T1]{fontenc}
\usepackage[utf8]{inputenc}
\usepackage{lmodern}
\usepackage[english, russian]{babel}
\usepackage{amsmath}
\usepackage{amsfonts}
\begin{document}

\section*{Лабораторная работа №6}

	При выполнении лабораторной работы обратите особое внимание на выбор оптимальных алгоритмов, типов данных. Проведите тестирование разработанных программ. Программы из задания номер 3 должны работать и при достаточно большом значении n, полученный результат проверить в Wolfram Alpha (https://www.wolframalpha.com/). Проверьте работоспособность разработанной программы из задания №5 (варианты 1-3,6,8,10, 12-16) при |x|>1. Полученные в задании № 5 результаты также проверьте в Wolfram Alpha.
	Отчёт по лабораторной работе выполняется в формате Markdown или TeX. Преподавателю предоставляется исходный файл отчёта, файл в формате pdf и работающие коды программы. В отчет по каждой задаче необходимо включить следующее: условие, блок-схема алгоритма, код программы, результаты тестирования программы.
\newpage
	\begin{center}
		
		\subsection*{Вариант 1}
	\end{center}


	\begin{enumerate} 
		\item Задание 1 \\ 
			Вывести на экран первые 100 чисел-палиндромов > 12.
		\item Задание 2\\
			Вводится последовательность целых чисел, которая заканчивается после ввода 13 простых чисел. 
			Для каждого введённого числа вывести на экран число, 
			которое получится после записи цифр исходного числа в обратном порядке.\\
		\item Задание 3 \\
			Найти сумму первых n  $( 100 \le n \le 1000) $ натуральных чисел, кратных 5.\\
		\item Задание 4 \\
			Составьте программу вычисления значения суммы  $S(x)=x-\frac{x^3}{3!}+...+(-1)^n\frac{x^{2n+1}}{(2n+1)!}$
			и функции $Y(x)=sin(x)$ в диапазоне от 0 до 1
			с произвольным шагом h. S(x) накапливать до тех пор, пока модуль очередного слагаемого не станет меньше $\epsilon$, вводимого с клавиатуры. Выведите на экран таблицу значений функции Y(x) и её разложение в ряд S(x). Близость значений Y(x) и S(x) во всём диапазоне
			значений x указывает на правильность их вычисления.\\
		\item Задание 5 \\
			Напишите программу для вычисления y по формуле:
			$$y=1-\frac{x^2}{2!}+\frac{x^4}{4!}-\frac{x^6}{6!}+...+(-1)^n \frac{x^{2n}}{(2n)!}.$$
			Натуральное значение n введите с клавиатуры. Значение x (|x|<1) также вводится с клавиатуры.\\
	\end{enumerate}

\newpage
	\begin{center}
		\subsection*{Вариант 2}
	\end{center}


	\begin{enumerate} 
		\item Задание 1 \\
			Вывести на экран первые 125 чисел-палиндромов > 100.\\
		\item Задание 2\\
			Вводится последовательность целых чисел, которая заканчивается после ввода n чисел-палиндромов.
			Найти наименьшее из введенных чисел, у которого сумма делителей нечётна.\\
		\item Задание 3 \\
			Найти произведение первых n простых чисел. Программа должна работать при n=15.\\
		\item Задание 4 \\
			Составьте программу вычисления значения суммы  $S(x)=1+\frac{x^2}{2!}+...+\frac{x^{2n}}{(2n)!}$
			и функции $Y(x)=\frac{e^x+e^{-x}}{2}$ в диапазоне от 0 до 1
			с произвольным шагом h. Значение n для вычисления суммы вводится с клавиатуры. Выведите на экран таблицу значений функции Y(x) и её разложение в ряд S(x). Близость значений Y(x) и S(x) во всём диапазоне
			значений x указывает на правильность их вычисления.\\
		\item Задание 5 \\
			Напишите программу для вычисления y по формуле:
			$$y=1+\frac{xlna}{1!}+\frac{(xlna)^2}{2!}+...+\frac{(xlna)^n}{n!}.$$
			Натуральное значение n введите с клавиатуры. Значения x (|x|<1) и а также вводятся с клавиатуры.\\
	\end{enumerate}
\newpage
	\begin{center}
		\subsection*{Вариант 3}
	\end{center}


	\begin{enumerate} 
		\item Задание 1 \\
			Вывести на экран последние 10 простых чисел в диапазоне 1000 до 10000.\\
		\item Задание 2\\
			Вводится последовательность целых чисел, которая заканчивается после ввода n чисел-палиндромов. 
			Найти два наименьших числа из введенных, у которых сумма делителей нечётна.\\
		\item Задание 3 \\
			Найти сумму первых n $(100 \le n \le 1000) $ палиндромов.\\
		\item Задание 4 \\
			Составьте программу вычисления значения суммы  $S(x)=1+\frac{cos(\frac{\pi}{4})}{1!}x+...+\frac{cos(n\frac{\pi}{4})}{n!}x^n$
			и функции $Y(x)=e^{xcos(\frac{\pi}{4})}cos(xsin\frac{\pi}{4})$ в диапазоне от 0 до 1
			с произвольным шагом h. S(x) накапливать до тех пор, пока модуль очередного слагаемого не станет меньше $\epsilon$, вводимого с клавиатуры. Выведите на экран таблицу значений функции Y(x) и её разложение в ряд S(x). Близость значений Y(x) и S(x) во всём диапазоне
			значений x указывает на правильность их вычисления.\\
		\item Задание 5 \\
			Напишите программу для вычисления y по формуле:\\
			$$y=x-\frac{x^2}{2}+\frac{x^3}{3}-\frac{x^4}{4}+...+(-1)^{n+1}\frac{x^n}{n}.$$\\
			Натуральное значение n введите с клавиатуры. Значение x (|x|<1) также вводятся с клавиатуры.\\
	\end{enumerate}
\newpage
	\begin{center}
		\subsection*{Вариант 4}
	\end{center}


	\begin{enumerate} 
		\item Задание 1 \\
			Вывести на экран первые 100 простых чисел > 25.\\
		\item Задание 2\\
			Вводится последовательность целых чисел, которая заканчивается после ввода n чисел-палиндромов.  
			Найти два наибольших числа из введенных, у которых сумма делителей чётна.\\
		\item Задание 3 \\
			Найти среднее арифметическое первых n $(100 \le n \le 1000) $ натуральных чисел, кратных 13, не кратных 2, 3, 5, 7.\\
		\item Задание 4 \\
			Составьте программу вычисления значения суммы  $S(x)=1-\frac{x^2}{2!}+...+(-1)^n\frac{x^{2n}}{(2n)!}$
			и функции $Y(x)=cos(x)$ в диапазоне от 0 до 1
			с произвольным шагом h. Значение n для вычисления суммы вводится с клавиатуры. Выведите на экран таблицу значений функции Y(x) и её разложение в ряд S(x). Близость значений Y(x) и S(x) во всём диапазоне
			значений x указывает на правильность их вычисления.\\
		\item Задание 5 \\
			Напишите программу для вычисления y по формуле:
			$$y=\frac{1}{2+\frac{1}{4+\frac{1}{6+...+\frac{1}{2n}}}}.$$
			Натуральное значение n введите с клавиатуры.\\
	\end{enumerate}
	\newpage
	\begin{center}
		\subsection*{Вариант 5}
	\end{center}


	\begin{enumerate} 
		\item Задание 1 \\
			Вывести на экран последние n простых чисел в диапазоне a до b.\\
		\item Задание 2\\
			Вводится последовательность целых чисел, которая заканчивается после ввода n чисел-палиндромов.  
			Введенные числа вывести в двоичной системе счисления.\\
		\item Задание 3 \\
			Найти квадрат суммы первых n $(100 \le n \le 1000) $ простых чисел.\\
		\item Задание 4 \\
			Составьте программу вычисления значения суммы  $S(x)=1+3x^2+...+\frac{2n+1}{n!}x^{2n}$
			и функции $Y(x)=(1+2x^2)e^{x^2}$ в диапазоне от 0 до 1
			с произвольным шагом h. S(x) накапливать до тех пор, пока модуль очередного слагаемого не станет меньше $\epsilon$, вводимого с клавиатуры. Выведите на экран таблицу значений функции Y(x) и её разложение в ряд S(x). Близость значений Y(x) и S(x) во всём диапазоне
			значений x указывает на правильность их вычисления.\\
		\item Задание 5 \\
			Напишите программу для вычисления y по формуле:
			$$y=1+\frac{1}{3+\frac{1}{5+\frac{1}{7+...+\frac{1}{2n+1}}}}.$$
			Натуральное значение n введите с клавиатуры.\\
	\end{enumerate}
	\newpage
	\begin{center}
		\subsection*{Вариант 6}
	\end{center}


	\begin{enumerate} 
		\item Задание 1 \\
			Вывести на экран первые n простых чисел в диапазоне a до b.\\
		\item Задание 2\\
			Вводится последовательность целых чисел, которая заканчивается после ввода n чисел-палиндромов.  
			Введенные числа вывести в восьмеричной системе счисления.\\
		\item Задание 3 \\
			Найти корень квадратный из суммы первых  n $(100 \le n \le 1000) $ чисел-палиндромов.\\
		\item Задание 4 \\
			Составьте программу вычисления значения суммы  $S(x)=1+3x^2+...+\frac{2n+1}{n!}x^{2n}$
			и функции $Y(x)=\frac{e^x-e^{-x}}{2}$ в диапазоне от 0 до 1
			с произвольным шагом h. Значение n для вычисления суммы вводится с клавиатуры. Выведите на экран таблицу значений функции Y(x) и её разложение в ряд S(x). Близость значений Y(x) и S(x) во всём диапазоне
			значений x указывает на правильность их вычисления.\\
		\item Задание 5 \\
			Напишите программу для вычисления y по формуле:\\
			$$y=x-\frac{x^2}{1\cdot2}+\frac{x^3}{2\cdot4}-\frac{x^4}{3\cdot8}+...+(-1)^{n+1}\frac{x^{n+1}}{n2^n}.$$\\
			Натуральное значение n введите с клавиатуры. Значение x (|x|<1) также вводятся с клавиатуры.\\
	\end{enumerate}
	\newpage
	\begin{center}
		\subsection*{Вариант 7}
	\end{center}


	\begin{enumerate} 
		\item Задание 1 \\
			Вывести на экран последние n чисел-палиндромов в диапазоне a до b.\\
		\item Задание 2\\
			Вводится последовательность целых чисел, которая заканчивается после ввода n чисел-палиндромов.
			 Вывести номер наибольшей цифры в каждом введенном числе.\\
		\item Задание 3 \\
			Найти сумму первых n $(100 \le n \le 1000) $ натуральных чисел, кратных 13 и 22.\\
		\item Задание 4 \\
			Составьте программу вычисления значения суммы  $S(x)=\frac{x^3}{3}-\frac{x^5}{15}+...+(-1)^{n+1}\frac{x^{2n+1}}{4n^2-1}$
			и функции $Y(x)=\frac{1+x^2}{2}arctg(-\frac{x}{2})$ в диапазоне от 0 до 1
			с произвольным шагом h.  S(x) накапливать до тех пор, пока модуль очередного слагаемого не станет меньше $\epsilon$, вводимого с клавиатуры. Выведите на экран таблицу значений функции Y(x) и её разложение в ряд S(x). Близость значений Y(x) и S(x) во всём диапазоне
			значений x указывает на правильность их вычисления.\\
		\item Задание 5 \\
			Напишите программу для вычисления y по формуле:
			$$y=\sqrt{1+\sqrt{3+\sqrt{5+...+\sqrt{2n+1}}}}.$$
			Натуральное значение n введите с клавиатуры.\\
	\end{enumerate}
	\newpage
	\begin{center}
		\subsection*{Вариант 8}
	\end{center}


	\begin{enumerate} 
		\item Задание 1 \\
			Вывести на экран первые n чисел-палиндромов в диапазоне a до b.\\
		\item Задание 2\\
			Вводится последовательность целых чисел, которая заканчивается после ввода n чисел-палиндромов. 
			Вывести номер наименьшей цифры в каждом введенном числе.\\
		\item Задание 3 \\
			Найти квадратный корень из суммы первых n $(100 \le n \le 1000) $ составных чисел.\\
		\item Задание 4 \\ 
			Составьте программу вычисления значения суммы  $S(x)=1+\frac{2x}{1!}+...+\frac{(2x)^n}{n!}$
			и функции $Y(x)=e^{2x}$ в диапазоне от 0 до 1
			с произвольным шагом h. Значение n для вычисления суммы вводится с клавиатуры. Выведите на экран таблицу значений функции Y(x) и её разложение в ряд S(x). Близость значений Y(x) и S(x) во всём диапазоне
			значений x указывает на правильность их вычисления.\\
		\item Задание 5 \\
			Напишите программу для вычисления y по формуле:\\
			$$y=1-\frac{3}{2}+\frac{3\cdot5}{2\cdot4}x^2-\frac{3\cdot5\cdot7}{2\cdot4\cdot6}x^3+...+(-1)^n\frac{3\cdot5\cdot...\cdot(2n+1)}{2\cdot4\cdot...\cdot2n}x^n.$$\\
			Натуральное значение n введите с клавиатуры. Значение x (|x|<1) также вводятся с клавиатуры.\\
	\end{enumerate}
	
	\newpage
	\begin{center}
		\subsection*{Вариант 9}
	\end{center}
	
	
	\begin{enumerate} 
		\item Задание 1 \\
		Вывести на экран первые n чисел-палиндромов > a.\\
		\item Задание 2\\
		Вводится последовательность целых чисел, которая заканчивается после ввода n чисел-палиндромов. 
		Найти количество чисел, состоящих из одинаковых цифр.\\
		\item Задание 3 \\
		Найти синус суммы первых первых n $(100 \le n \le 1000) $ составных чисел.\\
		\item Задание 4 \\ 
		Составьте программу вычисления значения суммы  $S(x)=1+2\frac{x}{2}+...+\frac{n^2+1}{n!}{(\frac{x}{2})^n}$
		и функции $Y(x)=(\frac{x^2}{2}+\frac{x}{2}+1)e^{\frac{x}{2}}$ в диапазоне от 0 до 1
		с произвольным шагом h. S(x) накапливать до тех пор, пока модуль очередного слагаемого не станет меньше $\epsilon$, вводимого с клавиатуры.  Выведите на экран таблицу значений функции Y(x) и её разложение в ряд S(x). Близость значений Y(x) и S(x) во всём диапазоне
		значений x указывает на правильность их вычисления.\\
		\item Задание 5 \\
		Напишите программу для вычисления y по формуле:\\
		$$y=\sqrt{2+\sqrt{4+\sqrt{6+...+\sqrt{2n}}}}.$$
		Натуральное значение n введите с клавиатуры.\\
	\end{enumerate}
\newpage
	\begin{center}
		\subsection*{Вариант 10}
	\end{center}
	
	
	\begin{enumerate} 
		\item Задание 1 \\
		Вывести на экран числа-палиндромы в диапазоне от a до b, их сумму и количество.\
		\item Задание 2\\
		Вводится последовательность целых чисел, которая заканчивается после ввода n чисел-палиндромов. Найти количество чисел, состоящих из нечётных цифр.\\
		\item Задание 3 \\
		Возвести произведение первых n простых чисел в степень m. Программа должна работать при $n \le 15$.\\
		\item Задание 4 \\ 
		Составьте программу вычисления значения суммы  $S(x)=x-\frac{x^3}{3}+...+(-1)^n\frac{x^{2n+1}}{2n+1}$
		и функции $Y(x)=arctg(x)$ в диапазоне от 0 до 1
		с произвольным шагом h. Значение n для вычисления суммы вводится с клавиатуры. Выведите на экран таблицу значений функции Y(x) и её разложение в ряд S(x). Близость значений Y(x) и S(x) во всём диапазоне
		значений x указывает на правильность их вычисления.\\
		\item Задание 5 \\
		Напишите программу для вычисления y по формуле:\\
		$$y=1-\frac{5}{2}x+\frac{5\cdot7}{2\cdot4}x^2-\frac{5\cdot7\cdot9}{2\cdot4\cdot6}x^3+...+(-1)^n\frac{5\cdot7\cdot...\cdot(2n+3)}{2\cdot4\cdot...\cdot2n}x^n.$$\\
		Натуральное значение n введите с клавиатуры. Значение x (|x|<1) также вводятся с клавиатуры.\\
	
	\end{enumerate}
\newpage
	\begin{center}
		\subsection*{Вариант 11}
	\end{center}
	
	
	\begin{enumerate} 
		\item Задание 1 \\
		Вывести на экран простые числа в диапазоне от a до b, их сумму и количество.\\
		\item Задание 2\\
		Вводится последовательность целых чисел, которая заканчивается после ввода n простых чисел. Для каждого введённого числа вывести все его делители.\\
		\item Задание 3 \\
		Найти корень кубический из суммы первых  n $(100 \le n \le 1000) $ простых чисел.\\
		\item Задание 4 \\ 
		Составьте программу вычисления значения суммы  $S(x)=1-\frac{3}{2}{x^2}+...+(-1)^n\frac{2n^2+1}{(2n)!}{x^{2n}}$
		и функции $Y(x)=(1-\frac{x^2}{2})cos(x)-\frac{x}{2}sin(x)$ в диапазоне от 0 до 1
		с произвольным шагом h. S(x) накапливать до тех пор, пока модуль очередного слагаемого не станет меньше $\epsilon$, вводимого с клавиатуры.  Выведите на экран таблицу значений функции Y(x) и её разложение в ряд S(x). Близость значений Y(x) и S(x) во всём диапазоне значений x указывает на правильность их вычисления.\\
		\item Задание 5 \\
		Напишите программу для вычисления y по формуле:\\
		$$y=\sqrt{2n+\sqrt{2(n-1)+...+\sqrt{4+\sqrt{2}}}}.$$
		Натуральное значение n введите с клавиатуры. \\
		
	\end{enumerate}
\newpage
	\begin{center}
		\subsection*{Вариант 12}
	\end{center}
	
	
	\begin{enumerate} 
		\item Задание 1 \\
		Среди чисел больших 100 найти первые 50 чисел-палиндромов и первые 70 простых чисел.\\
		\item Задание 2\\
		Вводится последовательность целых чисел, которая заканчивается после ввода n простых чисел. Для каждого введённого числа вывести сумму его делителей.\\
		\item Задание 3 \\
		Найти куб суммы первых n $(100 \le n \le 1000) $ чисел палиндромов.\\
		\item Задание 4 \\ 
		Составьте программу вычисления значения суммы  $S(x)=-\frac{(2x)^2}{2}+\frac{(2x)^4}{4}-...+(-1)^n\frac{(2x)^{2n}}{(2n)!}$
		и функции $Y(x)=2(cos{2x}-1)$ в диапазоне от 0 до 1
		с произвольным шагом h. Значение n для вычисления суммы вводится с клавиатуры. Выведите на экран таблицу значений функции Y(x) и её разложение в ряд S(x). Близость значений Y(x) и S(x) во всём диапазоне
		значений x указывает на правильность их вычисления.\\
		\item Задание 5 \\
		Напишите программу для вычисления y по формуле:\\
		$$y=1+\frac{x^2}{2!}+\frac{x^4}{4!}+\frac{x^6}{6!}+...+\frac{x^{2n}}{(2n)!}.$$\\
		Натуральное значение n введите с клавиатуры. Значение x (|x|<1) также вводятся с клавиатуры.\\
		
	\end{enumerate}
\newpage	
	\begin{center}
		\subsection*{Вариант 13}
	\end{center}
	
	
	\begin{enumerate} 
		\item Задание 1 \\
		Среди чисел больших a найти первые n чисел-палиндромов и первые m простых чисел.\\
		\item Задание 2\\
		Вводится последовательность целых чисел, которая заканчивается после ввода n простых чисел.  Для каждого введённого числа вывести его наибольший делитель, меньший самого числа.\\
		\item Задание 3 \\
		Найти сумму квадратов первых n $(100 \le n \le 1000) $ чисел, кратных 7.\\
		\item Задание 4 \\ 
		Составьте программу вычисления значения суммы  $S(x)=-{(1+x)}^2+\frac{{(1+x)}^4}{2}+...+(-1)^n\frac{{(1+x)}^{2n}}{n}$
		и функции $Y(x)=ln\frac{1}{2+2x+x^2}$ в диапазоне от 0 до 1
		с произвольным шагом h.  S(x) накапливать до тех пор, пока модуль очередного слагаемого не станет меньше $\epsilon$, вводимого с клавиатуры. Выведите на экран таблицу значений функции Y(x) и её разложение в ряд S(x). Близость значений Y(x) и S(x) во всём диапазоне
		значений x указывает на правильность их вычисления.\\
		\item Задание 5 \\
		Напишите программу для вычисления y по формуле:\\
		$$y=x+\frac{x^2}{1\cdot2}+\frac{x^3}{2\cdot4}+\frac{x^4}{3\cdot8}+...+\frac{x^{n+1}}{n2^n}.$$\\
		Натуральное значение n введите с клавиатуры. Значение x (|x|<1) также вводятся с клавиатуры.\\
		
	\end{enumerate}
\newpage
		\begin{center}
		\subsection*{Вариант 14}
	\end{center}
	
	
	\begin{enumerate} 
		\item Задание 1 \\
		Вводится последовательность целых чисел, которая заканчивается после ввода 20-го простого числа. Найти два наибольших значения из введенных чисел.\\
		\item Задание 2\\
		Вводится последовательность целых чисел, которая заканчивается после ввода n простых чисел. Для каждого введённого числа вывести его наименьший делитель >1.\\
		\item Задание 3 \\
		Найти корень пятнадцатой степени из произведения первых n чётных чисел, но не кратных 8. Программа должная работать при n=13\\
		\item Задание 4 \\ 
		Составьте программу вычисления значения суммы  $S(x)=\frac{x}{3}-\frac{x^3}{162}+...+(-1)^k\cdot\frac{{(\frac{x}{3})}^{2n+1}}{(2n+1)!}$
		и функции $Y(x)=sin(\frac{x}{3})$ в диапазоне от 0 до 1
		с произвольным шагом h. Значение n для вычисления суммы вводится с клавиатуры. Выведите на экран таблицу значений функции Y(x) и её разложение в ряд S(x). Близость значений Y(x) и S(x) во всём диапазоне
		значений x указывает на правильность их вычисления.\\
		\item Задание 5 \\
		Напишите программу для вычисления y по формуле:\\
		$$y=1+\frac{x^3}{3!}+\frac{x^5}{5!}+\frac{x^7}{7!}+...+\frac{x^{2n+3}}{(2n+3)!}.$$\\
		Натуральное значение n введите с клавиатуры. Значение x (|x|<1) также вводятся с клавиатуры.\\
		
	\end{enumerate}
	\newpage
	\begin{center}
		\subsection*{Вариант 15}
	\end{center}
	
	
	\begin{enumerate} 
		\item Задание 1 \\
		Вводится последовательность целых чисел, которая заканчивается после ввода 17-го числа-палиндрома. Найти наибольшее и наименьшее значения из введенных чисел.\\
		\item Задание 2\\
		Вводится последовательность целых чисел, которая заканчивается после ввода n простых чисел. Для каждого введённого числа проверить является ли сумма делителей числа простым числом.\\
		\item Задание 3 \\
		Найти  косинус суммы первых n $(100 \le n \le 1000) $ простых чисел.\\
		\item Задание 4 \\ 
		Составьте программу вычисления значения суммы  $S(x)=1-\frac{x^2}{18}+\frac{x^4}{1944}-...+(-\frac{1}{9})^n\cdot\frac{x^{2n}}{(2n)!}$
		и функции $Y(x)=cos(\frac{x}{3})$ в диапазоне от 0 до 1
		с произвольным шагом h.  S(x) накапливать до тех пор, пока модуль очередного слагаемого не станет меньше $\epsilon$, вводимого с клавиатуры. Выведите на экран таблицу значений функции Y(x) и её разложение в ряд S(x). Близость значений Y(x) и S(x) во всём диапазоне
		значений x указывает на правильность их вычисления.\\
		\item Задание 5 \\
		Напишите программу для вычисления y по формуле:\\
		$$y=-1+\frac{1+x}{2}-\frac{(1+x)^2}{4}+...+(-1)^{n+3}\frac{{(1+x)^n}}{2^n}.$$\\
		Натуральное значение n введите с клавиатуры. Значение x (|x|<1) также вводятся с клавиатуры.\\
		
	\end{enumerate}
	\newpage
	\begin{center}
		\subsection*{Вариант 16}
	\end{center}
	
	
	\begin{enumerate} 
		\item Задание 1 \\
		Вводится последовательность целых чисел, которая заканчивается после ввода 10 чисел-палиндромов. Вывести на экран все введенные простые числа, их сумму и произведение.\\
		\item Задание 2\\
		Вводится последовательность целых чисел, которая заканчивается после ввода n простых чисел. Найти наибольшее из введенных чисел, у которого сумма делителей чётна.\\
		\item Задание 3 \\
		Найти корень одиннадцатой степени из суммы первых n $(100 \le n \le 1000) $ простых чисел больших 150.\\
		\item Задание 4 \\ 
		Составьте программу вычисления значения суммы  $S(x)=1-\frac{x^2}{10}+\frac{x^4}{200}+...+\frac{(-x^2)^k}{10^k\cdot k!}$
		и функции $Y(x)=e^{-\frac{x^2}{10}}$ в диапазоне от 0 до 1
		с произвольным шагом h. Значение n для вычисления суммы вводится с клавиатуры. Выведите на экран таблицу значений функции Y(x) и её разложение в ряд S(x). Близость значений Y(x) и S(x) во всём диапазоне
		значений x указывает на правильность их вычисления.\\
		\item Задание 5 \\
		Напишите программу для вычисления y по формуле:\\
		$$y=(x+1)+\frac{(x+1)^3}{3!}+\frac{(x+1)^5}{5!}+\frac{(x+1)^7}{7!}+...+\frac{(x+1)^{2n+1}}{(2n+1)!}.$$
		Натуральное значение n введите с клавиатуры. Значение x (|x|<1) также вводятся с клавиатуры.\\
		
	\end{enumerate}
\end{document}
